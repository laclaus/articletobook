\documentclass{book}

\title{Test of book}
\usepackage{imakeidx}
\makeindex
\begin{document}
    \frontmatter
    \maketitle
    \tableofcontents
    \listoffigures
    \listoftables
    
    \part{Part I}
    \chapter{Test chap}
    \section{Test sec}
    \subsection{Test subsec}
    \subsubsection{Test subsubsec}
    \paragraph{Test para} Lorem ipsum dolor et amet
    \subparagraph{Test subpara}
    In this example, several keywords\index{keywords} will be used 
    which are important and deserve to appear in the Index\index{Index}.
    
    \chapter{Test chap 2}
    \section{Test sec 2}
    
    \mainmatter
    \chapter{Test chap 3}
    \section{Test sec 3}
    \clearpage
    Here we cite the texbook \cite{texbook}.
    \clearpage
    This page has the section in the header.
    
    \chapter{Test chap 4}
    \section{Test sec 4}
    
    \part*{Unnumbered part}
    \chapter*{Unnumbered chap}
    \section*{Unnumbered sec}
    \subsection*{Unnumbered subsec}
    \subsubsection*{Unnumbered subsubsec}
    \paragraph*{Unnumbered para} Lorem ipsum dolor et amet
    \subparagraph*{Unnumbered subpara}
    \clearpage
    Terms like generate\index{generate} and some\index{others} will also show up.
    \clearpage
    Terms in the index can also be nested \index{Index!nested}
    
    \appendix
    \chapter{Test chap 5}
    \section{Test sec 5}
    
    \backmatter
    \chapter{Test chap 6}
    \section{Test sec 6}
    \clearpage
    Nearly the end
    \clearpage
    The end
    \begin{thebibliography}{10}
        \bibitem{texbook}
        Donald E. Knuth (1986) \emph{The \TeX{} Book}, Addison-Wesley Professional.
    \end{thebibliography}
    \printindex
\end{document}