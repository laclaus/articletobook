% \iffalse meta-comment
%
% SVN keywords: $Revision$
%
% Copyright (C) 2022 by Sebastien Laclau
% -----------------------------------
%
% This file may be distributed and/or modified under the
% conditions of the LaTeX Project Public License, either version 1.3c
% of this license or (at your option) any later version.
% The latest version of this license is in:
%
% http://www.latex-project.org/lppl.txt
%
% and version 1.3c or later is part of all distributions of LaTeX
% version 2008 or later.
%
% \fi
%
% \iffalse
%
%\section{Identification}
%
%    This document class can only be used with \LaTeXe, so we make
%    sure that an appropriate message is displayed when another \TeX{}
%    format is used.
%
%    \begin{macrocode}
%<package>\NeedsTeXFormat{LaTeX2e}[2021/06/01]
%    \end{macrocode}
%
%    Announce the package name and its version:
%
%    \begin{macrocode}
%<package>\ProvidesPackage{articletobook}
%<*driver>
\ProvidesFile{articletobook.dtx}
%</driver>
[2022/02/10 v1.0.0
%<package> Converts article based classes to book based classes]
%<*package>
%</package>
%    \end{macrocode}
%
%\section{Driver}
%
% The next bit of code contains the documentation driver file for
% \TeX{}, i.e., the file that will produce the documentation you are
% currently reading. It will be extracted from this file by the
% {\sc docstrip} program.
%
%    \begin{macrocode}
%<*driver>
]
\documentclass{ltxdoc}
%    \end{macrocode}
%    Some things do not need indexing.
%    \begin{macrocode}
\DoNotIndex{\def, \gdef, \let, \newcommand, \NewDocumentCommand, \RenewDocumentCommand, \renewenvironment, \relax, \undef}
\DoNotIndex{\m@ne, \z@}
\DoNotIndex{\., \ ,\addvspace, \vspace, \hfill, \hspace, \nobreakspace, \par}
\DoNotIndex{\Huge, \huge, \Large, \large, \normalsize}
\DoNotIndex{\normalfont, \slshape}
\DoNotIndex{\newif, \cs{else}, \cs{fi}}
\DoNotIndex{\item}
%    \end{macrocode}
%    We do want an index, using line numbers, and a change log.
%    \begin{macrocode}
\EnableCrossrefs
\CodelineIndex
\setcounter{IndexColumns}{2}
\RecordChanges
%    \end{macrocode}
%    The following code retrieves the date and version information from 
%the file.
%    \begin{macrocode}
\GetFileInfo{articletobook.dtx}
%    \end{macrocode}
%    Here are some commonly used abbreviations:
%    \begin{macrocode}
% SVN keywords: $Revision: 91 $

\providecommand*{\Lopt}[1]{\textsf {#1}}         % typeset an option
\providecommand*{\file}[1]{\texttt {#1}}         % typeset a file
\providecommand*{\Lcount}[1]{\textsl {\small#1}} % typeset a counter
\providecommand*{\pstyle}[1]{\textsl {#1}}       % typeset a pagestyle
\providecommand*{\Lenv}[1]{\texttt {#1}}         % typeset an 
%environment
\providecommand*{\Lpack}[1]{\textsf {#1}}        % typeset a package
%    \end{macrocode}
%    We also want the full details.
%    \begin{macrocode}
\begin{document}
	\DocInput{articletobook.dtx}
	\PrintIndex
	\PrintChanges
\end{document}
%</driver>
%    \end{macrocode}
%
% \fi
%
% \CheckSum{0}
%
% \changes{v1.0.0-alpha}{2022/01/28}{Initial version}
% \changes{v1.0.0}{2022/02/10}{First release version}
%
%
%
% \title{The \Lpack{articletobook} package\thanks{This document
	% corresponds to \Lpack{articletobook}~\fileversion,
	% dated \filedate.}}
% \author{Sebastien Laclau \\ \texttt{slaclau@wellingtoncollege.org.uk} \\ \texttt{seb.laclau@gmail.com}}
%
% \maketitle
% \tableofcontents
%
% \StopEventually{}
%
% \section{The {\sc docstrip} modules}
%
% The following modules are used in the implementation to direct
% {\sc docstrip} in generating the external files:
% \begin{center}
	% \begin{tabular}{ll}
		%   package & produce the package \Lpack{articletobook}\\
		%   driver  & produce a documentation driver file \\
		% \end{tabular}
	% \end{center}
%
% \section{Initial code}
%
% \begin{macro}{\@ifmainmatter}
%    The switch |\@ifmainmatter| indicates whether we are processing 
%the main material in the book.
%    \begin{macrocode}
%<*package>
\newif\if@mainmatter \@mainmattertrue
%    \end{macrocode}
% \end{macro}
%
% \begin{macro}{\if@openright}
%    A switch to indicate if chapters must start on a right-hand page. 
%The default is yes.
%    \begin{macrocode}
\newif\if@openright \@openrighttrue
%    \end{macrocode}
% \end{macro}
%
% \begin{macro}{\if@twoside}
%    A switch to indicate if twosided printing is in use. The default is yes. It is defined already by \Lpack{article} so is only set here.
%    \begin{macrocode}
\@twosidetrue
%    \end{macrocode}
% \end{macro}
%
% \section{Declaration of options}
%
% \subsection{Options setup}
%
%    We use the package \Lpack{kvoptions} to provide a key value interface for options.
%    \begin{macrocode}
\RequirePackage{kvoptions}
\SetupKeyvalOptions{
    family = atob,
    prefix = atob@
}
%    \end{macrocode}
%
% \subsection{Options for disabling features which clash}
%
%    The option \Lopt{undefrules} is used to undefine |\headrule| and |\footrule| so that they may be defined subsequently by \Lpack{titlesec}.
%    \begin{macrocode}
\DeclareBoolOption[false]{undefrules}
%    \end{macrocode}
%
% \subsection{Options for correcting `features' of \Lpack{book}}
%
%    The option \Lopt{fixstarredmarks} sets chapter and section marks following starred chapters and sections.
%    \begin{macrocode}
\DeclareBoolOption[false]{fixstarredmarks}
%    \end{macrocode}
%
% \subsection{Options copied from the standard class \Lpack{book}}
%    These options determine whether or not a chapter must start on a right-hand page.
%    \begin{macrocode}
\DeclareVoidOption{openright}{\@openrighttrue}
\DeclareVoidOption{openany}{\@openrightfalse}
%    \end{macrocode}
%    These options determine whether two sided printing is in use or not.
%    \begin{macrocode}
\@twosidetrue
\DeclareVoidOption{twoside}{\@twosidetrue}
\DeclareVoidOption{oneside}{\@twosidefalse}
%    \end{macrocode}
%    These options determine whether to adjust the layout to match \Lpack{book}.
%    \begin{macrocode}
\DeclareBoolOption[true]{booklayout}
\DeclareComplementaryOption{originallayout}{booklayout}
%    \end{macrocode}
%
% \section{Executing options}

%    \begin{macrocode}
\ProcessKeyvalOptions*
\ifatob@booklayout
    \input{bk1\@ptsize.clo}
\fi

\ifatob@undefrules
    \undef{\headrule}
    \undef{\footrule}
\fi
%    \end{macrocode}
%
% \section{Loading packages}
%
%    This package makes heavy use of the \Lpack{titlesec} bundle of packages. It also used \Lpack{xparse} and \Lpack{etoolbox} to define and patch commands.
%    \begin{macrocode}
\RequirePackage[notocpart*,loadonly]{titlesec}
\RequirePackage{titletoc}
\RequirePackage[nopatches]{titleps}
\RequirePackage{xparse,etoolbox}
%    \end{macrocode}
%
% \section{Document layout}
%
% \subsection{Page styles}
%
% \subsubsection{Page style \pstyle{headings}}
%
% \begin{macro}{\ps@headings}
%    The definition of the page style \pstyle{headings} has to be different for two sided printing than it is for one sided printing.
%    \begin{macrocode}
\if@twoside
    \renewpagestyle{headings}{
%    \end{macrocode}
%    The running feet are empty in this page style, the running head contains the page number and one of the marks.
%    \begin{macrocode}
        \sethead[\thepage]%
                []%
                [\slshape\MakeUppercase{\ifnum\c@secnumdepth>\m@ne%
                    \if@mainmatter\if@noschapter%
                        \@chaptername\ \thechapter.\ %
                 \fi\fi\fi\bookchaptertitle}]%
        {\slshape\MakeUppercase{\ifnum\c@secnumdepth>\z@\if@nossection%
            \thesection.\ \fi\fi\sectiontitle}}%
        {}%
        {\thepage}
    }
%    \end{macrocode}
%    The definition of the page style \pstyle{headings} for one sided printing can be much simpler, because we treat even and odd pages the same.
%    \begin{macrocode}
\else
    \renewpagestyle{headings}{
        \sethead{\slshape\MakeUppercase{\ifnum\c@secnumdepth>\m@ne%
                    \if@mainmatter\@chaptername\ \thechapter.\ %
                    \fi\fi\bookchaptertitle}}%
                {}%
                {\thepage}
    }
\fi
%    \end{macrocode}
% \end{macro}
%
% \subsubsection{Page style \pstyle{myheadings}}
%
% \begin{macro}{\ps@myheadings}
%    The definition of the pagestyle \pstyle{myheadings} is simple as it is intended for customization.
%    \begin{macrocode}
\renewpagestyle{myheadings}{
    \renewcommand{\markright}[1]{\def\rightmark{##1}}
    \renewcommand{\markboth}[2]{\def\leftmark{##1}\def\rightmark{##2}}
    \sethead[\thepage]%
            []%
            [\slshape\leftmark]%
            {\slshape\rightmark}%
            {}%
            {\thepage}
}
%    \end{macrocode}
% \end{macro}
%
% \section{Document markup}
%
% \subsection{The title}
%
% \begin{macro}{\maketitle}
%    If a seperate titlepage is not being used then |\maketitle| is redefined to be empty -- an error results if this is not done. The reason for this is unclear.
%    \begin{macrocode}
\if@titlepage\else
\renewcommand{\maketitle}{}
\fi
%    \end{macrocode}
% \end{macro}
%
% \subsection{Defining counters}
%
% \begin{macro}{\c@secnumdepth}
%    The value of the counter \Lcount{secnumdepth} gives the depth of the highest-level sectioning command that is to produce section numbers.
%    \begin{macrocode}
\setcounter{secnumdepth}{2}
%    \end{macrocode}
% \end{macro}
% \begin{macro}{\c@bookpart}
% \begin{macro}{\c@bookchapter}
%    The value of these counters gives the current part or chapter. The section counter must be reset by the chapter counter using |\@addtoreset|.
%    \begin{macrocode}
\newcounter {bookpart}
\newcounter {bookchapter}
\@addtoreset{section}{bookchapter}
%    \end{macrocode}
% \end{macro}
% \end{macro}
% \begin{macro}{\thebookchapter}
% \begin{macro}{\thesection}
% \begin{macro}{\thesubsection}
% \begin{macro}{\thesubsubsection}
% \begin{macro}{\theparagraph}
% \begin{macro}{\thesubparagraph}
%    We redefine |\theCTR| for each section counter |\c@CTR|.
%    \begin{macrocode}
\renewcommand \thebookpart     {\@Roman\c@bookpart}
\renewcommand \thebookchapter  {\@arabic\c@bookchapter}
\renewcommand \thesection      {\thebookchapter.\@arabic\c@section}
\renewcommand \thesubsection   {\thesection.\@arabic\c@subsection}
\renewcommand \thesubsubsection{\thesubsection.\@arabic\c@subsubsection}
\renewcommand \theparagraph    {\thesubsubsection.\@arabic\c@paragraph}
\renewcommand \thesubparagraph {\theparagraph.\@arabic\c@subparagraph}
%    \end{macrocode}
% \end{macro}
% \end{macro}
% \end{macro}
% \end{macro}
% \end{macro}
% \end{macro}
%
% \subsection{Front Matter, Main Matter, and Back Matter}
%    A book contains these three (logical) sections. The switch |\@mainmatter| is true iff we are processing Main Matter. When this switch is false, the |\bookchapter| command does not print bookchapter numbers.
%
%    Here we define the commands that start these sections.
% \begin{macro}{\frontmatter}
%    This command starts Roman page numbering and turns off chapter numbering. Since this restarts the page numbering from 1, it should also ensure that a recto page is used.
%    \begin{macrocode}
\newcommand\frontmatter{%
    \if@openright
        \cleardoublepage
    \else
        \clearpage
    \fi
    \@mainmatterfalse
    \pagenumbering{roman}}
%    \end{macrocode}
% \end{macro}
% \begin{macro}{\mainmatter}
%    This command clears the page, starts arabic page numbering and turns on chapter numbering. Since this restarts the page numbering from 1, it should also ensure that a recto page is used.
%    \begin{macrocode}
\newcommand\mainmatter{%
    \if@openright
        \cleardoublepage
    \else
        \clearpage
    \fi
    \@mainmattertrue
    \pagenumbering{arabic}}
%    \end{macrocode}
% \end{macro}
% \begin{macro}{\backmatter}
%    This clears the page, turns off chapter numbering and leaves page numbering unchanged.
%    \begin{macrocode}
\newcommand\backmatter{%
    \if@openright
        \cleardoublepage
    \else
        \clearpage
    \fi
    \@mainmatterfalse}
%    \end{macrocode}
% \end{macro}
%
% \subsection{Parts, chapters, sections, and paragraphs}
%
% \begin{macro}{\bookpart}
% \begin{macro}{\bookchapter}
% \begin{macro}{\section}
% \begin{macro}{\subsection}
% \begin{macro}{\subsubsection}
% \begin{macro}{\paragraph}
% \begin{macro}{\subparagraph}
%    First, we redefine the sectioning structure of the class using |\titleclass| from \Lpack{titlesec}. We begin the sectioning at the part level using |\bookpart| internally. This will later be let to |\part|. Similarly, |\bookchapter| will be let to |\chapter|.
%    \begin{macrocode}
\titleclass{\bookpart}[-1]{page}
\titleclass{\bookchapter}{top}[\bookpart]
\titleclass{\section}{straight}[\bookchapter]
\titleclass{\subsection}{straight}[\section]
\titleclass{\subsubsection}{straight}[\subsection]
\titleclass{\paragraph}{straight}[\subsubsection]
\titleclass{\subparagraph}{straight}[\paragraph]
%    \end{macrocode}
%    We now format each of these sectioning commands in the same way as the standard class \Lpack{book}.
%    \begin{macrocode}
\titleformat{\bookpart}
    [display]
    {\centering\normalfont\huge\bfseries}
    {\partname\nobreakspace\thebookpart}
    {20pt}
    {\Huge}
\titlespacing*{\bookpart}{0pt}{0pt}{0pt}

\titleformat{\bookchapter}
    [display]
    {\normalfont\huge\bfseries\if@appendix%
        \gdef\@chaptername{\appendixname}\fi}
    {\@chaptername\ \thebookchapter}
    {20pt}
    {\Huge}
\titlespacing*{\bookchapter}{0pt}{50pt}{40pt}

\titleformat{\section}
    {\normalfont\Large\bfseries}{\thesection}{1em}{}
\titleformat{\subsection}
    {\normalfont\large\bfseries}{\thesubsection}{1em}{}
\titleformat{\subsubsection}
    {\normalfont\normalsize\bfseries}{\thesubsubsection}{1em}{}
\titleformat{\paragraph}[runin]
    {\normalfont\normalsize\bfseries}{\theparagraph}{1em}{}
\titleformat{\subparagraph}[runin]
    {\normalfont\normalsize\bfseries}{\thesubparagraph}{1em}{}

\titlespacing*{\chapter}
    {0pt}{50pt}{40pt}
\titlespacing*{\section}
    {0pt}{3.5ex plus 1ex minus .2ex}{2.3ex plus .2ex}
\titlespacing*{\subsection}
    {0pt}{3.25ex plus 1ex minus .2ex}{1.5ex plus .2ex}
\titlespacing*{\subsubsection}
    {0pt}{3.25ex plus 1ex minus .2ex}{1.5ex plus .2ex}
\titlespacing*{\paragraph}
    {0pt}{3.25ex plus 1ex minus .2ex}{1em}
\titlespacing*{\subparagraph}
    {\parindent}{3.25ex plus 1ex minus .2ex}{1em}
%    \end{macrocode}
%    Finally, we let |\bookpart| to |\part| and carefully define |\chapter| to account for the effects of |\@ifmainmatter|. We must also enable marks for |\bookchapter| and |\section|. We split |\chapter| and |\section| into starred and unstarred internal versions to allow for careful control of the marks.
%    \begin{macrocode}
\let\part\bookpart
\settitlemarks{bookchapter,section}
\newif\if@noschapter \@noschaptertrue
\NewDocumentCommand{\chapter}{s O{#3} m}{
    \IfBooleanTF{#1}{
        \@schapter{#2}{#3}
        \ifatob@fixstarredmarks\@noschapterfalse\else\@noschaptertrue\fi
    }{
        \@chapter{#2}{#3}
        \@noschaptertrue
    }
}
\NewDocumentCommand{\@schapter}{m m}{
    \bookchapter*{#2}
    
    \ifatob@fixstarredmarks\def\lastchaptermark{#1}\fi
    
    \chaptermark{\lastchaptermark}
}
\NewDocumentCommand{\@chapter}{m m}{
    \if@mainmatter
        \bookchapter[#1]{#2}
    \else
        \bookchapter*{#2}
        \addcontentsline{toc}{bookchapter}{#1}
    \fi
    \def\lastchaptermark{#1}
    \chaptermark{#1}
}
\let\oldsection\section
\newif\if@nossection \@nossectiontrue
\RenewDocumentCommand{\section}{s O{#3} m}{
    \IfBooleanTF{#1}{
        \@ssection{#2}{#3}
        \ifatob@fixstarredmarks\@nossectionfalse\else\@nossectiontrue\fi
    }{
        \@section{#2}{#3}
        \@nossectiontrue
    }
}
\NewDocumentCommand{\@ssection}{m m}{
    \oldsection*{#2}
    
    \ifatob@fixstarredmarks\def\lastsectionmark{#1}\fi
    
    \sectionmark{\lastsectionmark}
}
\NewDocumentCommand{\@section}{m m}{
    \oldsection[#1]{#2}
    
    \def\lastsectionmark{#1}
    \sectionmark{#1}
}
\newcommand{\chaptermark}[1]{\bookchaptermark{#1}}
\newcommand{\thechapter}{\thebookchapter}
%    \end{macrocode}
% \end{macro}
% \end{macro}
% \end{macro}
% \end{macro}
% \end{macro}
% \end{macro}
% \end{macro}
%
% \subsection{Redefining environments}
%
% \subsubsection{Abstract}
%
% \begin{environment}{abstract}
%    This environment should be undefined for \Lpack{book}
%    \begin{macrocode}
\undef{\abstract}
\undef{\endabstract}
%    \end{macrocode}
% \end{environment}
%
% \subsubsection{Titlepage}
%
% \begin{environment}{titlepage}
%    This environment includes a |\cleardoublepage| which is not present in \Lpack{article}
%    \begin{macrocode}
\renewenvironment{titlepage}
    {%
        \cleardoublepage
        \if@twocolumn
        \@restonecoltrue\onecolumn
        \else
        \@restonecolfalse\newpage
        \fi
        \thispagestyle{empty}%
        \setcounter{page}\z@
    }%
    {\if@restonecol\twocolumn \else \newpage \fi
    }
%    \end{macrocode}
% \end{environment}
%
% \subsubsection{Equation and eqnarray}
%
% \begin{macro}{\theequation}
%    We need the equation counter to reset with the \Lcount{bookchapter} counter and we also need to redefine |\theequation|.
%    \begin{macrocode}
\@addtoreset{equation}{bookchapter}
\renewcommand{\theequation}{\ifnum \c@bookchapter>\z@%
    \thebookchapter.\fi \@arabic\c@equation}
%    \end{macrocode}
% \end{macro}
%
% \subsubsection{Appendix}
%
% \begin{macro}{\appendix}
%    We redefine the |\appendix| command to reset the bookchapter and section counters, start producing alphabetic chapter numbers, and set |\@ifappendix|. This switch is used to redefine |\@chaptername| to |\appendixname| ensuring that this change occurs in the right place.
%    \begin{macrocode}
\newif\if@appendix \@appendixfalse
\renewcommand{\appendix}{
    \par
    \setcounter{bookchapter}{0}
    \setcounter{section}{0}
    \@appendixtrue
    \gdef\thebookchapter{\@Alph\c@bookchapter}
}
%    \end{macrocode}
% \end{macro}
%
% \subsection{Floating objects}
%
% \subsubsection{Figure}
%
% \begin{macro}{\c@figure}
%    We need the \Lcount{figure} counter to reset with the \Lcount{bookchapter} counter and we also need to redefine |\thefigure|.
%    \begin{macrocode}
\@addtoreset{figure}{bookchapter}
\renewcommand{\thefigure}{\ifnum \c@bookchapter>\z@%
    \thebookchapter.\fi \@arabic\c@figure}
%    \end{macrocode}
% \end{macro}
%
% \subsubsection{Table}
%
% \begin{macro}{\c@table}
%    We need the \Lcount{table} counter to reset with the \Lcount{bookchapter} counter and we also need to redefine |\thetable|.
%    \begin{macrocode}
\@addtoreset{table}{bookchapter}
\renewcommand{\thetable}{\ifnum \c@bookchapter>\z@%
    \thebookchapter.\fi \@arabic\c@table}
%    \end{macrocode}
% \end{macro}
%
% \section{Cross referencing}
%
% \subsection{Table of contents etc.}
%
% \begin{macro}{\c@stocdepth}
%    The value of the counter \Lcount{tocdepth} gives the depth of the highest-level sectioning command that is to produce section numbers in the table of contents.
%    \begin{macrocode}
\setcounter{tocdepth}{2}
%    \end{macrocode}
% \end{macro}
%
% \subsubsection{Table of contents}
%
% \begin{macro}{\tableofcontents}
%    We need to redefine the table of contents formatting using |\titlecontents|
%    \begin{macrocode}
\renewcommand\tableofcontents{%
    \if@twocolumn
        \@restonecoltrue\onecolumn
    \else
        \@restonecolfalse
    \fi
    \chapter*{\contentsname}%
    \chaptermark{\contentsname}
    \vspace{-40pt}
    \@starttoc{toc}%
    \if@restonecol\twocolumn\fi
}

\contentsmargin{2.55em}

\titlecontents{bookpart}
    [1.6em]
    {\addvspace{3.25em}\bfseries\large}
    {\contentslabel{1.4em}}
    {\hspace*{0em}}
    {\normalsize\hfill\contentspage{\large}}

\titlecontents{bookchapter}
    [1.5em]
    {\addvspace{1em}\bfseries}
    {\contentslabel{1.5em}}
    {\hspace*{-1.5em}}
    {\titlerule*[1pc]{}\contentspage}

\titlecontents{section}
    [3.8em]
    {}
    {\contentslabel{2.3em}}
    {\hspace*{-1.5em}}
    {\titlerule*[1pc]{.}\contentspage}

\titlecontents{subsection}
    [7.1em]
    {}
    {\contentslabel{3.2em}}
    {\hspace*{-2.3em}}
    {\titlerule*[1pc]{.}\contentspage}

\titlecontents{subsubsection}
    [9.3em]
    {}
    {\contentslabel{4.1em}}
    {\hspace*{-3.2em}}
    {\titlerule*[1pc]{.}\contentspage}

\titlecontents{paragraph}
    [12.3em]
    {}
    {\contentslabel{5em}}
    {\hspace*{-4.1em}}
    {\titlerule*[1pc]{.}\contentspage}

\titlecontents{subparagraph}
    [14.3em]
    {}
    {\contentslabel{6em}}
    {\hspace*{-5em}}
    {\titlerule*[1pc]{.}\contentspage}
%    \end{macrocode}
% \end{macro}
%
% \subsubsection{List of figures}
%
% \begin{macro}{\listoffigures}
%    We need the list of figures to begin with |\chapter*|.
%    \begin{macrocode}
\renewcommand\listoffigures{%
    \if@twocolumn
        \@restonecoltrue\onecolumn
    \else
        \@restonecolfalse
    \fi
    \chapter*{\listfigurename}%
    \chaptermark{\listfigurename}
    \vspace{-40pt}
    \@starttoc{lof}%
    \if@restonecol\twocolumn\fi
}
%    \end{macrocode}
% \end{macro}
%
% \subsubsection{List of tables}
%
% \begin{macro}{\listoftables}
%    We need the list of tables to begin with |\chapter*|.
%    \begin{macrocode}
\renewcommand\listoftables{%
    \if@twocolumn
        \@restonecoltrue\onecolumn
    \else
        \@restonecolfalse
    \fi
    \chapter*{\listtablename}%
    \chaptermark{\listtablename}
    \vspace{-40pt}
    \@starttoc{lot}%
    \if@restonecol\twocolumn\fi
}
%    \end{macrocode}
% \end{macro}
%
% \subsection{Bibliography}
%
% \begin{environment}{thebibliography}
%    We need the bibliography to begin with |\chapter*|.
%    \begin{macrocode}
\renewenvironment{thebibliography}[1]
{\chapter*{\bibname}%
    \chaptermark{\bibname}
    \list{\@biblabel{\@arabic\c@enumiv}}%
    {\settowidth\labelwidth{\@biblabel{#1}}%
        \leftmargin\labelwidth
        \advance\leftmargin\labelsep
        \@openbib@code
        \usecounter{enumiv}%
        \let\p@enumiv\@empty
        \renewcommand\theenumiv{\@arabic\c@enumiv}}%
    \sloppy
    \clubpenalty4000
    \@clubpenalty \clubpenalty
    \widowpenalty4000%
    \sfcode`\.\@m}
{\def\@noitemerr
    {\@latex@warning{Empty `thebibliography' environment}}%
    \endlist}
%    \end{macrocode}
% \end{environment}
%
% \subsection{The index}
%
% \begin{environment}{theindex}
%    We need the index to begin with |\chapter*|.
%    \begin{macrocode}
\renewenvironment{theindex}
{\if@twocolumn
    \@restonecolfalse
    \else
    \@restonecoltrue
    \fi
    \twocolumn[\chapter*{\indexname}\chaptermark{\indexname}]%
    \thispagestyle{plain}\parindent\z@
    \parskip\z@ \@plus .3\p@\relax
    \columnseprule \z@
    \columnsep 35\p@
    \let\item\@idxitem}
{\if@restonecol\onecolumn\else\clearpage\fi}
%    \end{macrocode}
% \end{environment}
%
% \section{Initialization}
%
%    Finally, we initialize a few necessary commands.
%
%    \begin{macrocode}
\def\lastsectionmark{}
\def\lastchaptermark{}
\newcommand{\bibname}{Bibliography}
\newcommand{\chaptername}{Chapter}
\newcommand{\@chaptername}{\chaptername}
\pagestyle{headings}
%</package>
%    \end{macrocode}
% \Finale
\endinput